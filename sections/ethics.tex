\section{Ethical Issues}
\label{sec:ethics}
EMBERS, as an anticipatory intelligence system, has many powerful legitimate uses but is also
susceptible to abuse.

First, it is important to have a discussion of civil unrest and its role in society.
In our opinion, under the proper circumstances, 
civil unrest enhances the ability of citizens to communicate not only their views but also
their priorities to those who govern them. Governments constantly need to make choices and find
it difficult to know, on specific issues at particular times, how their constituencies value the available
options. Elections are retrospective indicators and rarely issue-specific; polling taps into sentiment,
but is not a good indicator of priorities or strength of feeling because of the
low cost associated with responding. Events, on the other hand, indicate a willingness to
bear some costs (organization, mobilization, identification) in support of an issue and
thus reveal not only preferences but provide some indication of priorities.

An open sources indicators approach, as we have used here, is a potentially powerful
tool for understanding the social construction of meaning and its translation into behavior.
EMBERS can contribute to making the transmission of citizen preferences to
government less costly to the economy and society as well. There are economic
costs to even peaceful disruptions embodied in civil unrest due to lost work
hours and the deployment of police to manage traffic and the interactions
between protestors and bystanders. Given the vulnerability of large gatherings
to provocation by handfuls of violence-oriented protestors (e.g., Black Box
anarchists in Brazil) the economic, social and political costs of
large-scale public demonstrations are also potentially significant to marchers, bystanders, property owners
and the government -- democratically elected or not. The right to demonstrate can still
be respected but if the government responds to grievances in time,
the protestors may cancel the event or fewer people might participate in the event.
In today's interconnected society, protests also cause disruptions to supply chain logistics, travel, and
other sectors, and anticipating disruptions is key to ensuring safety as well as reliability.

The potential power of civil unrest forecasting systems, like those
of most scientific advances, is susceptible to abuse by
both democratic and non-democratic governments. The appropriate safeguards require
developing transparent and accountable democratic systems, not outlawing science.
Non-democratic governments may clearly abuse such forecasting systems. But even
here the value of forecasting civil unrest is not simply negative. Many non-democratic regimes transition to democratic
ones, often in a violent process but not always (in Latin America, authoritarian regimes negotiated
transitions to democracy without a civil war in Mexico, Honduras, Peru, Bolivia, Brazil,
Uruguay, Argentina and Chile). The rational choice models of authoritarian decision-making in
such crises always explain a dictatorship’s collapse rather than accommodation to
a transition by pointing to the lack of credible information in
a dictatorship regarding citizens’ true feelings. EMBERS-like models may thus provide the
information that facilitates and encourages some transitions.

