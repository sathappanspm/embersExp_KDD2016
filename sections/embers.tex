\section{Background}
We begin by providing a brief review of forecasting systems, followed by a
quick preview of EMBERS, its system architecture, machine learning models,
and measures for evaluating its performance. For more details, please see~\cite{kdd-beating}.

Forecasting societal events such as civil unrest has a long tradition in the intelligence analysis and political science
community. We distinguish between forecasting systems versus event coding systems (that provide
structured representations of ongoing events reported in newspapers), and focus on the former.
Early forecasting systems such as ICEWS~\cite{icews} provided very broad coverage in countries but
were limited by their spatio-temporal resolution (e.g., typically country- and month- level forecasting for
specific events of interest~\cite{eoiprediction}). These events of interest are
domestic political crises, international crises, ethnic/religious violence, insurgencies,
and rebellion.
A similar project in scope is PITF (Political Instability
Task Force)~\cite{pitf} funded by the CIA.
To the best of our knowledge, only EMBERS provides
the most specific spatial resolution (city-level) and temporal resolution (daily-level) capability in forecasting. 

The software architecture of EMBERS (Early Model Based Event Recognition using Surrogates) is designed 
\narenc{Patrick complete this using your own words.}

The structure of a civil unrest forecast is shown below:
\narenc{Give an example of a forecast here. With arrows.}
As shown here, a forecast constitutes four fields, corresponding to the when, where, who, and why
of the protest. These fields are respectively denoted as the date, location, population, and event type.
Location is recorded at the city level. Population and event type are fields chosen from a categorical
set of possibilities. 

Rather than design one model to integrate all possible data sources, EMBERS adopted a multi-model 
approach to forecasting. Each model utilized a specific (possibly overlapping) set of data sources
and is tuned for high precision, so that the union of these models can be tuned for high recall.
A fusion/suppression engine~\cite{andy-scotland-paper} allows a tunable strategy to issue fewer/more
alerts depending on whether the analyst's objective is higher precision versus recall. The
underlying models used in EMBERS are: (i) planned protest model~\cite{pp-paper1,pp-paper2},
(ii) dynamic query expnasion~\cite{dqe-plosone}, (iii) volume-based model~\cite{asonam},
and (iv) a baseline model. The planned protest model, for news and social media (Twitter, Facebook),
identifies explicit signs of organization and calls
for protest, resolves relative mentions of time (e.g., `next Saturday') and space (e.g., `the square')
to issue forecasts. The dynamic query expansion (DQE) model uses Twitter as a data source and learns time- and country-specific
expansions of a seed set of keywords to identify specific situational circumstances for civil unrest.
For instance, in Venezuela (an economy where the government exercises stringent price controls),
there were a series of protests in 2014 stemming from the shortage of toilet paper, a novel circumstance
that was uncovered by DQE. The volume-based model uses a range of data sources, spanning
social, economic
and political indicators. It uses clasical statistical models (Lasso
and hybrid regression models) to forecast civil unrest events using features
from social media (Twitter, blogs), news sources, 
political event databases (ICEWS and GDELT~\cite{gdelt}), Tor~\cite{tor} statistics, food prices, and currency
exchange rates. It aims to provide a multi-source perspective into forecasting by leveraging
the selective superiorities of different data sources. Finally, the baseline model uses maximum likelihood
estimation over the GSR to issue history-based forecasts.

The EMBERS project is unique not just in its algorithmic underpinnings but also in the use of new measures
for evaluation, specifically aimed at determining forecasting performance. As shown in Fig.~\ref{leadtime},
one of the primary measures of EMBERS performance is lead time, the number of days by which a forecast
`beats the news', i.e., the date of reporting of the event. Lead time should not be confused with 
date quality, i.e., the difference between the predicted date and the actual date of the event. The date
quality is one of the components to the quality score, the other components being the
location score, event type score, and population score. Fig.~\ref{otherscores} show how these other
components are scored between an EMBERS forecast and a GSR record. Given a set of alerts and a set of
events for a given month, the lead time is used as a constraint to define legal (alert, event) pairs so that
we can construct a bipartite matching to optimize the best quality score. From this bipartite matching,
measures of precision and recall can be derived, i.e., by assessing the number of (un)matched events or
alerts. Finally, a confidence score is used to assess the quality of probabilities imputed by EMBERS to its
forecasts, and measured in terms of the Brier score. For more details, please see~\cite{beating-embers-kdd}. 

We now turn to a discussion of specific discoveries enabled by EMBERS, into civil unrest in Latin America, and into
the complexity of the forecasting enterprise, as a whole.

