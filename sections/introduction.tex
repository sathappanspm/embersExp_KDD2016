\section{Introduction}
Modern communication forms such as social media and microblogs are not only rapidly
advancing our understanding of the world but also improving the methods by which we can
comprehend, and even forecast, the progression of events.
Tracking population-level activities via `massive passive' data has been shown to
quite accurately shed light into large-scale societal movements.

Two years ago, in KDD 2014, we described EMBERS~\cite{kdd:beating-the-news}, a deployed anticipatory
intelligence system~\cite{bigdata-andy-doyle-embers-paper} that forecasts significant
societal events (e.g., civil unrest
events such as protests, strikes, and `occupy' events) using a large set of open source
indicators such as news, blogs, tweets, food prices, currency rates, and other public
data. The EMBERS system has been running continuously 24x7 for nearly 4 years at this point
and our goal in this paper is to present the discoveries it has enabled,
both correct as well as missed
forecasts, and lessons learned from participating in a forecasting tournament including
our perspectives on the limits of forecasting and ethical considerations. In
particular, we shed insight into the value proposition to an analyst and how EMBERS forecasts
are communicated to its end-users.

The development of EMBERS is supported by the Intelligence Advanced Research Projects
Activity (IARPA) Open Source Indicators (OSI) program.
EMBERS forecasts are scored against the Gold Standard Report (GSR), a monthly catalog of
events as reported in newspapers of record in 10 Latin American
countries - Argentina, Brazil, Chile, Colombia, Ecuador, El Salvador,
Mexico, Paraguay, Uruguay, Venezuela. The GSR is compiled by MITRE corporation
using human analysts.
EMBERS currently focuses on multiple regions of the world but for the purpose of this paper
we focus primarily on Latin America, specifically the countries of
Argentina, Brazil, Chile, Colombia, Ecuador, El Salvador, Mexico, Paraguay, Uruguay, and Venezuela.
Similarly, EMBERS generates forecasts for multiple event classes---influenza like illnesses~\cite{prithwish-ili},
rare diseases~\cite{sdm-saurav}, elections~\cite{aravindan-wei-besc}, domestic political crises~\cite{gdelt-acm-webscience}, and civil unrest---but in this paper we focus primarily on civil unrest as this was the
most challenging event class with hundreds of events every month across the countries studied here.

Our key contributions can be summarized as follows:
%\narenc{Refer to sections in each bullet.}
\begin{enumerate}
\setlength\itemsep{0pt}
\itemsep0em
\item % (Section~\ref{sec:perf})
  Unlike retrospective studies of predictability, EMBERS forecasts are communicated in real-time before the
event to MITRE/IARPA and scored independently of the authors.
We present multiple quantitative indicators of EMBERS performance
as well as insights into how
we made EMBERS forecasts most valuable to analysts. We report two primary
ways in which analysts utilize EMBERS and the use of {\it automated
narratives} to help make EMBERS forecasts as useful as possible.

\item
In an attempt to demystify the state-of-the-art
in forecasting and to create an open dialogue in the community, we report both successful
forecasts of EMBERS as well as events missed by EMBERS. The events not forecast by EMBERS lead us to
considerations of both the limitations of the underlying technology as well as
the inherent limits to forecasting large-scale events.
\item
While social media is often touted as the key to event forecasting
systems such as EMBERS, we present the results of an ablation study to outline
the performance degradation that ensues if data sources
such as Twitter and Facebook were to be removed from the forecasting pipeline.
\item
We consider the separation of civil unrest events into events
that happen with a degree of regularity versus rare
or significant
events, and evaluate the performance of EMBERS in forecasting such
surprising events.
\item
We describe our current best understanding of the limitations to
forecasting civil unrest events using technologies like EMBERS and
also consider the ethical considerations of the EMBERS technologies.
\end{enumerate}

