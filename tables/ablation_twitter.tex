%\begin{table}
%\caption{Comparison of performance metrics with and without social media sources. Social Media sources contributes a lot towards recall but loses out on
%lead-time.}
%\label{tb:ablation_twitter}
%\begin{tabular}{|c|c|c|c|c|}
%\hline
%Data Source          & Quality-Score & Lead-time & Precision & Recall \\
%\hline
%Social Media Sources & \cellcolor[red]{.16} -16.48\% & \cellcolor[red]{.55} -55\% & \cellcolor[green]{.35} +35\% & \cellcolor[red]{.14} -14\% \\
%%Social Media Sources & \cellcolor[red]{.16} -16.48\% & \cellcolor{red!55} -55\% & \cellcolor{green!35} +35\% & \cellcolor{red!14} -14\% \\
%\hline
%Non-Social Media Sources & \cellcolor{green!10} +8.42\% & \cellcolor{green!40} +30\%  & \cellcolor{green!60} +79\% & \cellcolor{red!30} -33\% \\
%\hline
%\end{tabular}
%\end{table}
\begin{table}
%\small
\caption{Comparison of performance measures under ablation testing.
Social media sources contribute toward recall but, due to their noisy
nature, lower other measures of performance.}
\label{tb:ablation_twitter}
\vspace{-3mm}
\resizebox{\columnwidth}{!}{
\begin{tabular}{|L|c|c|c|c|}
\hline
Data Source          & Quality-Score & Lead-time & Precision & Recall \\
\hline
Removing news and blogs & -16.48\% &-55\% &+35\% &-14\% \\
\hline
Removing social media & +8.42\% & +30\%  &+79\% &-33\% \\
\hline
\end{tabular}
}
\end{table}
